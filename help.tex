% arara: xelatex

\documentclass[12pt]{article}

\usepackage{listings}

\usepackage{graphicx}
\usepackage{multicol}
\setlength{\columnsep}{25pt}

% Márgenes y espaciado
\usepackage[a4paper, margin=2.5cm]{geometry}

\usepackage[spanish]{babel} % Idioma del documento: Español

% Fuente
\usepackage{fontspec} % Permite utilizar fuentes TrueType y OpenType
\setmainfont{Linux Libertine}

% Colores
\usepackage{xcolor}
\definecolor{myorange}{RGB}{214,93,14}
\definecolor{myblue}{RGB}{0,102,204}

% Encabezado y pie de página
\usepackage{fancyhdr}
\pagestyle{fancy}
\fancyhf{}
\lhead{\leftmark}
\rhead{{\bfseries\thepage}}
\renewcommand{\headrulewidth}{2pt}
\renewcommand{\footrulewidth}{0pt}

% Títulos de sección
\usepackage{titlesec}
\titleformat{\section}
	{\normalfont\Large\bfseries\color{myorange}}
	{\thesection}{1em}{}
\titlespacing*{\section}{0pt}{3.5ex plus 1ex minus .2ex}{2.3ex plus .2ex}

% Notas de colores
\usepackage{tcolorbox}
\tcbuselibrary{skins}
\newtcolorbox{mybox}[1]{colback=myorange!5!white,colframe=myorange!75!black,fonttitle=\bfseries,title=#1}

% Enlaces clicables
\usepackage[colorlinks=true, linkcolor=myorange, urlcolor=myorange]{hyperref}

\title{\textcolor{myorange}{\bfseries Configuración de OpenBSD}} % Título del documento
\date{\today} % Fecha actual
\author{aleister888}

\begin{document}

\maketitle

\section{Introducción}

Este documento pretende ser una guía para terminar de configurar nuestra instalación de OpenBSD, pues hay cosas que no son factibles auto-configurar. Así como puede ser una ayuda para terceros que se han animado a usar mi instalador y que ahora se encuentran ante la tarea de aprender a usar un entorno de trabajo que no es el suyo.

\begin{mybox}{Nota}
Cualquier texto que veas coloreado en naranja es un hiper-vínculo \emph{(exceptuando los títulos)}, si haces doble clic en él, te llevará a la página web a la que hace referencia o a una donde se explica de que estoy hablando.
\end{mybox}

\section{Atajos de teclado}

\href{https://dwm.suckless.org}{Dwm} es el programa que se encarga de repartir el espacio de trabajo entre nuestras ventanas \emph{(aplicaciones gráficas)}. Hay una cantidad enorme de \href{https://www.youtube.com/results?search_query=dwm}{videos sobre dwm}, como funciona y sus ventajas. Aquí veremos simplemente los atajos de teclado que tengo configurados:

\begin{itemize}
\setlength\itemsep{-0.4em}
\item Alt Izq. + Ctrl + S: Abrir este documento
\item Alt Izq. + P: Abrir lanzador de comandos
\item Alt Izq. + Shift + P: Abrir lanzador de aplicaciones
\item Alt Izq. + Shift + Intro: Abrir terminal
\begin{itemize}
\setlength\itemsep{-0.3em}
\item Alt Izq. + C: Copiar texto
\item Alt Izq. + V: Pegar texto
\item Alt Izq. + L: Abrir link en el navegador
\end{itemize}
\item Alt Izq. + F1 / Alt + Shift + F1: Configurar pantallas
\item Alt Izq. + F2: Abrir Navegador \emph{(Firefox)}
\item Alt Izq. + F3: Abrir Administrador de archivos
\item Alt Izq. + Shift + F3: Abrir Administrador de archivos (En /run/media, donde se montan los pendrives y discos externos)
\begin{itemize}
\setlength\itemsep{-0.3em}
\item Espacio: Seleccionar archivos
\item Shift + S: Borrar archivos
\item Shift + D: Mover archivos a la papelera
\item Ctrl + D: Vaciar papelera
\item Alt Izq. + D: Restaurar papelera
\item Shift + P: Mirar el tamaño de una carpeta
\item Ctrl + Z: Permitir arrastrar archivos a otra ventana
\item D: Cortar archivos
\item Y: Copiar archivos
\item P: Pegar archivos
\item Shift + E: Extraer archivo
\item Ctrl + E: Comprimir contenidos de la carpeta actual
\item Ctrl + R: Renombrar los contenidos de la carpeta actual
\item R: Renombrar archivo
\item S: Abrir shell
\end{itemize}
\item Alt Izq. + F4: Abrir el reproductor de música
\item Alt Izq. + F5: Montar dispositivo android en el árbol de ficheros
\item Alt Izq. + Shift + F5: Desmontar dispositivo android del árbol de ficheros
\item Alt Izq. + F11: Abrir gestión de energía
\item Alt Izq. + Shift + F11: Reiniciar dwm
\item Alt Izq. + F12: Abrir mezclador de sonido
\item Alt Izq. + Z: Canción anterior
\item Alt Izq. + X: Canción siguiente
\item Alt Izq. + Shift + Z/X: Pausar/reanudar la reproducción
\item Alt Izq. + N: Bajar Volumen
\item Alt Izq. + M: Subir Volumen
\item Alt Izq. + Shift + N: Establecer volumen al 50\%
\item Alt Izq. + Shift + M: Establecer volumen al 100\%
\item Alt Izq. + Ctrl + N/M: Silenciar/activar el sonido
\item Alt Izq. + O: Captura de pantalla al portapapeles
\item Alt Izq. + Ctrl + O: Guardar captura de pantalla
\item Alt Izq. + Shift + O: Captura de un área de la pantalla al porta-papeles
\item Alt Izq. + Ctrl + Shift + O: Guardar captura de un área de la pantalla
\item Alt Izq. + B: Ocultar/Mostrar la barra de estado
\item Alt Izq. + ,: Mover el foco a la posición anterior
\item Alt Izq. + Shift + ,: Mover la ventana a la posición anterior
\item Alt Izq. + .: Mover el foco a la posición siguiente
\item Alt Izq. + Shift + .: Mover la ventana a la posición siguiente
\item Alt Izq. + Q: Moverse al espacio anterior
\item Alt Izq. + W: Moverse al espacio siguiente
\item Alt Izq. + 1-9: Moverse al espacio 1-9
\item Alt Izq. + Shift + 1-9: Mover ventana al espacio 1-9
\item Alt Izq. + Shift + Q: Cerrar ventana
\item Alt Izq. + \{: Mover el foco al monitor anterior
\item Alt Izq. + \}: Mover el foco al monitor siguiente
\item Alt Izq. + Shift + \}: Mover ventana al monitor anterior
\item Alt Izq. + Shift + \{: Mover ventana al monitor siguiente
\item Alt Izq. + F: Abrir \emph{terminal scratchpad}
\item Alt Izq. + S: Mostrar/Ocultar \emph{scratchpads}
\item Alt Izq. + Shift + S: Marcar/desmarcar la ventana seleccionada como \emph{strachpad}
\item Alt Izq. + Ctrl + S: Marcar/desmarcar la ventana seleccionada como \emph{sticky}
\item Alt Izq. + Ctrl + Clic Izq.: Mover ventana
\item Alt Izq. + Ctrl + Clic Der.: Re-dimensionar ventana
\end{itemize}

\section{Cronie y script útiles}

Mi instalación viene con varios scripts para automatizar tareas pero que por defecto no se usan para nada.

Con tu instalación base deberías de tener activado el servicio \href{https://wiki.archlinux.org/title/cron}{crond}. Este servicio se encarga de ejecutar comandos de forma automática. El archivo de configuración se encuentra en \emph{/etc/crontab} y tal como se ha configurado, debería verse como:

\begin{lstlisting}[basicstyle=\scriptsize\ttfamily]
SHELL=/bin/sh
MAILTO=usuario

# Auto suspend laptop
* * * * *	usuario		$HOME/.local/bin/bat
* * * * *	root		rm /var/log/Xorg.*
* * * * *	root		rm /var/log/daemon.*
* * * * *	root		rm /var/log/maillog.*
30 10 * * *	root		cd /usr/ports && cvs -q up -Pd -A
\end{lstlisting}

\begin{itemize}
\setlength\itemsep{-0.2em}
\item \emph{"30 10 * * *"} es la parte de donde definimos cuando se ejecutara el comando, si quieres investigar hay un montón de guías en youtube o cualquier otra plataforma sobre cual es la sintáxis para configurar la ejecución de los comandos. Puedes también comprobar si tu sintaxis es correcta en \href{https://crontab.guru/}{crontab.guru}.
\item \emph{"root"} es el usuario que ejecuta el comando, y \emph{cd /usr/ports \&\& cvs -q up -Pd -A}, el comando.
\end{itemize}

\subsection{convert-2m4a y convert-2mp3}

Este script coge toda la música del directorio que damos como primer agumento y nos hace un mirror en la carpeta que damos como segundo argumento en formato \emph{m4a} o \emph{mp3}.

Si ejecutamos:
\begin{verbatim}
convert-2mp3 /musica/biblioteca /musica/mp3
\end{verbatim}

esto nos convertirá toda la música de \emph{/musica/biblioteca} a la carpeta \emph{/musica/mp3}, en formato mp3. Lo que nos puede resultar de interés porque los archivos mp3 ocupan menos que su equivalente en flac u otros formatos sin pérdida de calidad. Personalmente, uso este script para convertir toda mi música a una carpeta que se sincroniza con mi móvil, que no tiene almacenamiento suficiente para guardar mi biblioteca música en su calidad original.

Podemos automatizar este proceso, para que el usuario \emph{"usuario1"} convierta a mp3 su música, todos los dias a las 8:30, añadiendo esta linea a \emph{/etc/crontab}:

\begin{verbatim}
30 8 * * * usuario1 convert-2mp3 /musica/biblioteca /musica/mp3
\end{verbatim}

\subsection{corruption-check}

Este script comprueba que no haya archivos corruptos en nuestra biblioteca de música, corrige falsos positivos de corrupción y nos escribe una lista con los archivos que realmente están corruptos y no se pueden reproducir correctamente en \emph{/tmp/corruption.log}. Solo necesita como argumento el directorio cuyos archivos de audio queremos comprobar.

Podemos automatizar esta tarea añadiendo el comando a \emph{/etc/crontab}, aquí un ejemplo:

\begin{verbatim}
15 7 * * * usuario1 corruption-check /musica/biblioteca
\end{verbatim}

\subsection{exif-remove}

Este script necesita como argumento un directorio y borrará toda la información \href{https://en.wikipedia.org/wiki/Exif}{EXIF} de las imágenes que contiene el mismo.

Los metadatos EXIF sirven para identificar el usuario que tomo la fotografía, por ejemplo, si tomas una fotografía con tu teléfono, el teléfono guarda como información EXIF datos como; el teléfono desde el cual la fotografía fue tomada, o desde que coordenadas geográficas fue tomada la foto. Esto es útil para compartir imágenes guardadas en tu ordenador reduciendo la huella digital que dejas cuando las compartes.

Para borrar automaticamente los metadatos de una carpeta, puedes añadir a \emph{/etc/crontab} una línea parecida a:

\begin{verbatim}
0 17 * * * usuario1 exif-remove /fotografias
\end{verbatim}

\subsection{Limpiar cache periódicamente}

Puede interesarte limpiar periódicamente archivos de registro y cache antigua de tu ordenador. Para borrarla de forma periódica para el usuario \emph{usuario1} añade a tu \emph{/etc/crontab}:

\begin{verbatim}
30 7 * * */2 usuario1 find ~/.cache -mtime +2 -delete
30 7 * * */2 usuario1 find ~/ -name "*.log" -mtime +4 -delete
\end{verbatim}

\subsection{compressed-backup}

Este script crea un fichero comprimido \emph{tar.gz} con una copia de seguridad del directorio que se le da por primer argumento en el directorio que se le da por segundo argumento. Además se encarga de borrar las copias de seguridad que tienen mas de un mes automáticamente.

Por ejemplo, añadir esta linea a nuestro \emph{/etc/crontab}:

\begin{verbatim}
0 16 * * 3 usuario1 compressed-backup ~/Música /Resguardos/Música
\end{verbatim}

Nos crearía todos los miércoles a las 16:00 una copia de seguridad de la carpeta \emph{~/Música} en un archivo comprimido en \emph{/Resguardos/Música}

\subsection{Copias de seguridad}

Podemos usar el comando \emph{rsync} para cerar copias de seguridad tipo mirror de nuestro directorio /home en otro disco duro.

Por ejemplo, añadir esta línea a nuestro \emph{/etc/crontab}:

\begin{verbatim}
0 16 * * */2 usuario1 rsync -av --delete /home/$(whoami)/ /mnt/Mirror1
0 16 * * */7 usuario1 rsync -av --delete /home/$(whoami)/ /mnt/Mirror2
\end{verbatim}

Hará que se cree un mirror de nuestro directorio \emph{/home} en \emph{/mnt/Mirror1} cada 2 días y otro en \emph{/mnt/Mirror2} cada semana.

\section{SSH}

Si queremos conectarnos a un equipo remoto a través de internet debemos configurar \href{https://es.wikipedia.org/wiki/OpenSSH}{OpenSSH}. Por defecto incluyo un script que configura SSH de forma muy básica, configurando SSH para poder acceder a tu equipo remoto con una contraseña. Para activar SSH simplemente ejecuta:

\begin{verbatim}
ssh-configure
\end{verbatim}

Mi recomendación es no conformarse con esta configuración básica. Desactiva el login con el usuario root, desactiva el login por contraseña y usa claves públicas para conectarte. Si no sabes de lo que estoy hablando, es mejor que no uses SSH y no te expongas a abrir la puerta a tu ordenador al internet, aunque sea con candado.

\subsection{VNC através de SSH}

Podemos usar SSH para acceder a nuestro entorno gráfico de forma remota y usar nuestro ordenador sin estar necesariamente delante de el físicamente. Por defecto dwm inicia un servidor \href{https://en.wikipedia.org/wiki/Virtual_Network_Computing}{VNC} para que puedas conectarte remotamente a una interfaz gráfica. Para poder hacer uso del servidor VNC tienes que usar \href{https://en.wikipedia.org/wiki/Tunneling_protocol}{tunneling}. Para conectarte y poder usar VNC, ejecuta:

\begin{verbatim}
ssh usuario1@255.255.255.255 -L 5900:localhost:5900
\end{verbatim}

Sustituye \emph{usuario1} por el usuario de tu máquina, y \emph{255.255.255.255} por la dirección IP de tu máquina.

Para conectarte a tu torre debes tener instalado algún cliente VNC, mi recomendación es \href{https://github.com/FreeRDP/Remmina}{remmina}. Abre tu cliente VNC y conéctate a \emph{localhost:5900}, eso es todo.

\section{Monitores de altas tasas de refresco}

Por defecto dwm actualiza el movimiento de las ventanas a 60 FPS. Si tienes un monitor de mas de 60 Hz esto puede hacer la experiencia de usar dwm insatisfactoria. Para cambiar este comportamiento, debes editar el archivo \emph{dwm.c}. En el encontrarás varias funciones con esta misma linea de código:

\begin{verbatim}
if ((ev.xmotion.time - lasttime) <= (1000 / 60))
\end{verbatim}

Deberás cambiar el valor \emph{60} por la tasa de refresco de tu monitor en cada aparición de esta linea. Por ejemplo, para un monitor de \emph{144 Hz}, la líneas a cambiar deberían verse así:

\begin{verbatim}
if ((ev.xmotion.time - lasttime) <= (1000 / 144))
\end{verbatim}

\section{Chromium}

\subsection{Instalar extensiones}

\end{document}
